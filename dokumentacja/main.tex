\documentclass{article}
\usepackage{graphicx} % Required for inserting images
\usepackage[T1]{fontenc}
\usepackage[polish]{babel}
\graphicspath{ {.} }

\title{dokumentacja Jimp 2025 JAVA}
\author{Kamil Gomuła, Artur Arciszewski}

\begin{document}

\maketitle



\section{Cel programu }

Załadowanie i wizualizacja grafu oraz umożliwienie podzielenia tego grafu w taki sposób, aby zachować margines jednocześnie przecinając możliwie najmniej połączeń. Graf jest zapisany w pliku w formacie \textit{csrrg.}

\section{Użycie programu: }

Kompilowanie i odpalanie programu różni się w zależności od programu operacyjnego maszyny użytkownika
\begin{itemize}
    \item Windows

w terminalu należy wpisać make.bat a potem start.bat aby uruchomić aplikację
\end{itemize}

\begin{itemize}
    \item  Linux

    należy w terminalu wpisać make compile a następnie make run aby uruchomić nasz program
\end{itemize}

\begin{figure}
    \centering
    \includegraphics[width=0.85\linewidth]{Ekran aplikacji z przykładowym grafem.png}
    \caption{Ekran aplikacji z przykładowym grafem}
    \label{fig:enter-label}
\end{figure}


Po uruchomieniu pojawi się okno aplikacji, gdzie na górze okna są odpowiednio:

przycisk do wyszukania pliku wejściowego, marginesu dla ciecia grafu (liczba zostaje przekształcona na procenty zatem 1 to 100\%), ilość cięć jakie są do wykonania, nazwa pliku wyjściowego (domyślnie output.txt) oraz przycisk tnący graf i zapisujący go do pliku.

Oddzielne grafy z jednego pliku są rozróżniane kolorami na interfejsie graficznym.


\section{Opis plików wejściowych}

\subsection{plik .csrrg}
Plik csrrg składa się z wielu linii. Poniżej opisane jest znaczenie poszczególnych linii:
\begin{enumerate}
    \item Maksymalna możliwa liczba węzłów w wierszu (w grafie nie musi znajdować się wiersz o takiej liczbie węzłów)
    \item Indeksy węzłów w poszczególnych wierszach - liczba wszystkich indeksów odpowiada liczbie węzłów grafu
    \item Wskaźniki na pierwsze indeksy węzłów w liście wierszy z punktu 2
    \item Grupy węzłów połączone przy pomocy krawędzi
    \item Wskaźniki na pierwsze węzły w poszczególnych grupach z punktu 4. Ta sekcja może występować w pliku wielokrotnie, co oznacza, że plik zawiera więcej niż jeden graf.
\end{enumerate}
\subsection{plik .out}
    plik .out jest plikiem wyjściowym z programu ze strony \textit{https://github.com/SBQD-nng/Dzielenie-Grafu}

\section{Opis plików wyjściowy}
plik wyjściowy jest zapisywany w formacie .csrrg (opis formatu w seckcji 3.1)


\section{Implementacja}
Po klinując przycisku \textit{Przetnij i Zapisz} program wykona podaną liczę cięć. \\
Program każde cięcie traktuje oddzielnie, każdy podział utworzy dokładnie dwa grafy których rozmiar spełnia poniży wzór:
\[\frac{|b-a|}{\frac{a+b}{2}}<q\] Gdzie \textbf{a} i \textbf{b} to rozmiary nowo powstałych grafów a \textbf{q} to margines podawany przy wywołaniu \\
Przykładowo jeśli kazał byś graf o 8 wierzchołkach podzielić 3 razy z marginesem 0.1 to otrzymach grafy o ramiarach 2, 2 i 4

\section{Zastosowana metoda podziału}
Do podziału grafu jest wykorzystany algorytm Stoer–Wagner aby zapewnić możliwie najmniejsza liczę przerwanych krawędzi. W przypadku, kiedy podział się nie uda graf pozostanie niezmieniony. \\
Algorytm Stoer–Wagner posiada pewną dozę losowości w efekcie dla tych samych warunków początkowych może być uzyskany inny wynik
\section{Link do programu}
https://github.com/PwAAstud/JIMP\_Projekt\_2025\_JAVA.git
\end{document}
